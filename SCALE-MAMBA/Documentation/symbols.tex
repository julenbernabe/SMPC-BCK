\newcommand{\sshare}[1]{\langle #1 \rangle}    
\newcommand{\bsshare}[1]{\Big\langle #1 \Big\rangle}  
\newcommand{\BList}{\mathsf{BitsList}}
\newcommand{\SList}{\mathsf{SquaresList}}
\newcommand{\MList}{\mathsf{MultList}}
\newcommand{\Zk}{\Z_{\langle k \rangle}}
\newcommand{\Zgam}{\Z_{\langle \gamma \rangle}}
\newcommand{\Qk}[1]{\Q_{\langle k, #1 \rangle}}
\newcommand{\Random}{\mathsf{Random}}
\newcommand{\Open}{{\mathsf{Open}}}
\newcommand{\Mult}{\mathsf{Mult}}

\newcommand{\abort}{\mathsf{abort}}
\newcommand{\asn}{\leftarrow}
\mathchardef\mhyphen="2D

\newcommand{\F}{{\mathbb{F}}}
\newcommand{\N}{{\mathbb{N}}}
\newcommand{\Q}{{\mathbb{Q}}}
\newcommand{\Z}{{\mathbb{Z}}}

\newcommand{\true}{{\mathsf{true}}}
\newcommand{\false}{{\mathsf{false}}}

\newcommand{\floor}[1]{{\lfloor #1 \rfloor}}
\newcommand{\ceil}[1]{{\lceil #1 \rceil}}
\newcommand{\angles}[1]{{\langle #1 \rangle}}

\newcommand{\err}{\mathsf{err}}

\newcommand{\modp}{\bmod \; p}
\newcommand{\modn}{\bmod \; 2^{64}}
\newcommand{\modtwo}{\bmod \; 2}

\newcommand{\note}[2]{{\textcolor{blue}{\sf (#1's Note:) {\sl{#2}}}}}
\newcommand{\todo}[1]{{\textcolor{red}{\sf (TO DO:) {\sl{#1}}}}}
\newcommand{\notetoreader}[1]{{\bf \em Note}: #1}





%\usepackage[small,nohug,heads=vee]{diagrams}
%\diagramstyle[labelstyle=\scriptstyle]
\titleformat*{\paragraph}{\normalfont\normalsize\bfseries\color{blue}}

\lstset{ %
language=Python,                % choose the language of the code
basicstyle=\ttfamily\small,       % the size of the fonts that are used for the code
numbers=none,                   % where to put the line-numbers
numberstyle=\tiny,      % the size of the fonts that are used for the line-numbers
stepnumber=1,                   % the step between two line-numbers. If it is 1 each line will be numbered
numbersep=5pt,                  % how far the line-numbers are from the code
%backgroundcolor=none,  % choose the background color. You must add \usepackage{color}
showspaces=false,               % show spaces adding particular underscores
showstringspaces=false,         % underline spaces within strings
showtabs=false,                 % show tabs within strings adding particular underscores
frame=none,           % adds a frame around the code
tabsize=2,          % sets default tabsize to 2 spaces
captionpos=b,           % sets the caption-position to bottom
breaklines=true,        % sets automatic line breaking
breakatwhitespace=false,    % sets if automatic breaks should only happen at whitespace
escapeinside={\%*}{*)}          % if you want to add a comment within your code
}







\lstnewenvironment{mylisting}{%
  \vspace{0.5em}
}{%
  \vspace{0.8em}
}

\newenvironment{class}[1]
{ {\noindent\colorbox{black!10}{\texttt{#1}}}\hspace{0pt} \begin{quote} }
{ \end{quote} }

\newcommand{\displaytt}[1]{\vspace{1em}\centerline{\texttt{#1}}\vspace{1em}}

\newcommand{\func}[1]{\paragraph{\texttt{#1}}\hspace{0pt} \vspace{0.5em}\\}

\newcommand{\id}{\mathfrak{i}}

\newcommand{\Red}[1]{{\color{red}{#1}}}
\newcommand{\Blue}[1]{{\color{blue}{#1}}}

\newcommand{\grp}[1]{{\langle #1 \rangle}}

\mathchardef\mhyphen="2D

\newcommand{\Sqr}{\mathsf{Sqr}}
\newcommand{\Ran}{\mathsf{Ran}}
\newcommand{\SolvedBits}{\mathsf{Solved{\mhyphen}Bits}}
\newcommand{\Symm}{\mathsf{Symm}}
\newcommand{\Pre}{\mathsf{Pre}}
\newcommand{\BitLT}{\mathsf{Bit{\mhyphen}LT}}
\newcommand{\BitAdd}{\mathsf{Bit{\mhyphen}Add}}
\newcommand{\Xor}{\mathsf{Xor}}

\newcommand{\ModTm}{\mathsf{Mod2m}}
\newcommand{\Trunc}{\mathsf{Trunc}}
\newcommand{\TruncPr}{\mathsf{TruncPr}}
\newcommand{\Bits}{\mathsf{Bits}}
\newcommand{\Sgn}{\mathsf{Sgn}}
\newcommand{\LTZ}{\mathsf{LTZ}}
\newcommand{\GTZ}{\mathsf{GTZ}}
\newcommand{\EQZ}{\mathsf{EQZ}}
\newcommand{\LT}{\mathsf{LT}}
\newcommand{\GE}{\mathsf{GE}}
\newcommand{\EQ}{\mathsf{EQ}}
\newcommand{\Scale}{\mathsf{Scale}}
\newcommand{\FPAdd}{\mathsf{FP{\mhyphen}Add}}
\newcommand{\FPMult}{\mathsf{FP{\mhyphen}Mult}}

\newcommand{\ow}{\overline{w}}
\newcommand{\ox}{\overline{x}}
\newcommand{\oy}{\overline{y}}
\newcommand{\oz}{\overline{z}}

\newcommand{\open}{\text{open}}
\newcommand{\startopen}{\mathtt{startopen}}
\newcommand{\stopopen}{\mathtt{stopopen}}
\newcommand{\triple}{\mathtt{triple}}
\newcommand{\squ}{\mathtt{square}}
\newcommand{\bit}{\mathtt{bit}}
\newcommand{\inp}{\mathtt{input}}
\newcommand{\inv}{\mathtt{inv}}



\mathchardef\mhyphen="2D

\newcommand{\ind}{\stackrel{c}{\approx}}


\newcommand{\C}{\mathbb{C}}
\newcommand{\R}{\mathbb{R}}

\newcommand{\gc}{\mathfrak{c}}
\newcommand{\ct}{\mathfrak{c}}
\newcommand{\xt}{\mathfrak{d}}
\newcommand{\sk}{\mathfrak{s}}
\newcommand{\pk}{\mathfrak{pk}}
\newcommand{\epk}{\mathfrak{epk}}
\newcommand{\lev}{\ell}

\newcommand{\enc}{\mathfrak{enc}}
\newcommand{\zero}{\mathfrak{zero}}

\newcommand{\va}{\mathbf{a}}
\newcommand{\vb}{\mathbf{b}}
\newcommand{\vc}{\mathbf{c}}
\newcommand{\vct}{\underline{\mathbf{c}}}
\newcommand{\vd}{\mathbf{d}}
\newcommand{\ve}{\mathbf{e}}
\newcommand{\vf}{\mathbf{f}}
\newcommand{\vk}{\mathbf{k}}
\newcommand{\vm}{\mathbf{m}}
\newcommand{\vr}{\mathbf{r}}
\newcommand{\vs}{\mathbf{s}}
\newcommand{\vt}{\mathbf{t}}
\newcommand{\vv}{\mathbf{v}}
\newcommand{\vx}{\mathbf{x}}
\newcommand{\vy}{\mathbf{y}}
\newcommand{\vz}{\mathbf{z}}


\newcommand{\norm}[1]{{\Vert #1 \Vert}}
\newcommand{\can}{\mathsf{can}}
\newcommand{\DDsecp}{\mathsf{DD\_sec}}
\newcommand{\Soundsecp}{\mathsf{Sound\_sec}}
\newcommand{\ZKsecp}{\mathsf{ZK\_sec}}
\newcommand{\Sacsecp}{\mathsf{sacrifice\_sec}}
\newcommand{\Macsecp}{\mathsf{macs\_sec}}

\newcommand{\Sample}{\mathsf{Sample}}
\newcommand{\comm}{\mathsf{comm}}
\newcommand{\FRand}{\mathcal{F}_{\mathsf{Rand}}}

\newcommand{\KeyGen}{\mathsf{KeyGen}}
\newcommand{\Enc}{\mathsf{Enc}}
\newcommand{\Dec}{\mathsf{Dec}}
\newcommand{\DistDec}{\mathsf{DistDec}}
\newcommand{\SwitchModulus}{\mathsf{SwitchModulus}}
\newcommand{\SwitchKey}{\mathsf{SwitchKey}}


\newcommand{\Seed}{\mathsf{Seed}}
\newcommand{\PRF}{\mathsf{PRF}}
\newcommand{\cnt}{\mathsf{cnt}}
\newcommand{\cond}{\mathsf{cond}}

\newcommand{\Rand}{\mathsf{Rand}}
\newcommand{\MACCheck}{\mathsf{MACCheck}}
\newcommand{\EncCommit}{\mathsf{EncCommit}}
\newcommand{\Reshare}{\mathsf{Reshare}}
\newcommand{\DataGen}{\mathsf{DataGen}}
\newcommand{\DataCheck}{\mathsf{DataCheck}}
\newcommand{\MACKeyGen}{\mathsf{MACKeyGen}}

\newcommand{\clean}{\mathsf{clean}}
\newcommand{\scale}{\mathsf{scale}}
\newcommand{\dec}{\mathsf{dec}}
\newcommand{\dishonest}{\mathsf{dishonest}}
\newcommand{\KS}{\mathsf{KS}}

\newcommand{\Var}{\mathsf{Var}}

\newcommand{\chall}{\mathsf{chall}}

\newcommand{\RLWE}{\mathsf{RLWE}}

\newcommand{\dN}{\mathcal{DG}}
\newcommand{\HWT}{\mathcal{HWT}}
\newcommand{\ZO}{\mathcal{ZO}}
\newcommand{\RC}{\mathcal{RC}}
\newcommand{\calU}{\mathcal{U}}

\newcommand{\cH}{\mathcal{H}}
\newcommand{\Comm}{\mathsf{Commit}}
\newcommand{\Challenge}{\mathsf{Chal}}
\newcommand{\Resp}{\mathsf{Response}}
\newcommand{\Verify}{\mathsf{Verify}}
\newcommand{\Deliver}{\mathsf{Deliver}}
\newcommand{\Abort}{\mathsf{Abort}}
\newcommand{\Skip}{\mathsf{Skip}}
\newcommand{\Cheat}{\mathsf{Cheat}}
\newcommand{\Proceed}{\mathsf{Proceed}}
\newcommand{\Pass}{\mathsf{Pass}}
\newcommand{\Continue}{\mathsf{Continue}}
\newcommand{\CheatSwitch}{\mathsf{CheatSwitch}}
\newcommand{\Success}{\mathsf{Success}}
\newcommand{\NoSuccess}{\mathsf{NoSuccess}}
\newcommand{\Start}{\mathsf{Start}}
\newcommand{\Command}{\mathsf{Command}}
\newcommand{\DataType}{\mathsf{DataType}}
\newcommand{\InputPrep}{\mathsf{InputPrep}}
\newcommand{\Triples}{\mathsf{Triples}}
\newcommand{\Squares}{\mathsf{Squares}}
\newcommand{\Angle}{\mathsf{Angle}}
\newcommand{\SEncCommit}{\mathsf{SEncCommit}}
\newcommand{\SReshare}{\mathsf{SReshare}}
\newcommand{\BreakDown}{\mathsf{BreakDown}}
\newcommand{\Decision}{\mathsf{Decision}}
\newcommand{\resp}{\mathsf{resp}}

\newcommand{\Func}{\mathcal{F}}
\newcommand{\Sim}{\mathcal{S}}
\newcommand{\Env}{\mathcal{Z}}
\newcommand{\Adv}{\mathcal{A}}

\newcommand{\FKEYGEN}{\ensuremath{\mathcal{F}_\textsc{KeyGen}}}
\newcommand{\PIKEYGEN}{\ensuremath{\Pi_\textsc{KeyGen}}}
\newcommand{\SKEYGEN}{\ensuremath{\mathcal{S}_\textsc{KeyGen}}}
\newcommand{\FCOMM}{\ensuremath{\mathcal{F}_\textsc{Commit}}}
\newcommand{\PICOMM}{\ensuremath{\Pi_\textsc{Commit}}}
\newcommand{\SCOMM}{\ensuremath{\mathcal{S}_\textsc{Commit}}}
\newcommand{\FENCCOMMIT}{\ensuremath{\mathcal{F}_\textsc{EncCommit}}}
\newcommand{\PIENCCOMMIT}{\ensuremath{\Pi_\textsc{EncCommit}}}
\newcommand{\SENCCOMMIT}{\ensuremath{\mathcal{S}_\textsc{EncCommit}}}
\newcommand{\FCOMMOPEN}{\ensuremath{\mathcal{F}_\textsc{Comm}}}
\newcommand{\FMACCHECK}{\ensuremath{\mathcal{F}_\textsc{MACCheck}}}
\newcommand{\SMACCHECK}{\ensuremath{\mathcal{S}_\textsc{MACCheck}}}
\newcommand{\FPREP}{\ensuremath{\mathcal{F}_\textsc{Prep}}}
\newcommand{\FMACKEYGEN}{\ensuremath{\mathcal{F}_\textsc{MACKeyGen}}}
\newcommand{\SMACKEYGEN}{\ensuremath{\mathcal{S}_\textsc{MACKeyGen}}}
\newcommand{\STUPLEGEN}{\ensuremath{\mathcal{S}_\textsc{TupleGen}}}
\newcommand{\STUPLECHECK}{\ensuremath{\mathcal{S}_\textsc{TupleCheck}}}
\newcommand{\SPREP}{\ensuremath{\mathcal{S}_\textsc{Prep}}}
\newcommand{\FSHE}{\ensuremath{\mathcal{F}_\textsc{SHE}}}
\newcommand{\PISHE}{\ensuremath{\Pi_\textsc{SHE}}}
\newcommand{\SSHE}{\ensuremath{\mathcal{S}_\textsc{SHE}}}
\newcommand{\Online}{\ensuremath{\Pi_\textsc{Online}}}
\newcommand{\FONLINE}{\ensuremath{\mathcal{F}_\textsc{Online}}}
\newcommand{\SONLINE}{\ensuremath{\mathcal{S}_\textsc{Online}}}
\newcommand{\Prep}{\ensuremath{\Pi_\textsc{Prep}}}

\newcommand{\Proto}[1]{\Pi_\mathsf{#1}}
\newcommand{\Simu}[1]{\mathcal{S}_{\mathsf{#1}}}

\newcommand{\Pgpok}{\Proto{gZKPoK}}
\newcommand{\Sgpok}{\Simu{gZKPoK}^S}
\newcommand{\gZKPoK}{\mathsf{ZKPoK}}
\newcommand{\flag}{\mathsf{flag}}
\newcommand{\Pred}{\mathsf{P}}
\newcommand{\Diag}{\mathsf{Diag}}

\newcommand{\ui}{^{(i)}}
\newcommand{\uti}{\ui{}^{\intercal}}
\newcommand{\transpose}[1]{#1^{\intercal}}

\def\tick{\tikz\fill[scale=0.4](0,.35) -- (.25,0) -- (1,.7) -- (.25,.15) -- cycle;}

\newcommand{\figlab}[1]{\label{fig:#1}}
\newcommand{\figref}[1]{Figure~\ref{fig:#1}}
\newcommand{\tablab}[1]{\label{tab:#1}}
\newcommand{\tabref}[1]{Table~\ref{tab:#1}}

% Environments
\newenvironment{boxfig}[2]{% {#1}{#2} = {Caption}{label}
   \vspace{0.3cm}  
     \begin{figure}[htbp!]
     \renewcommand{\figurename}{{\bf Figure~}}
     \newcommand{\FigCaption}{#1}
     \newcommand{\FigLabel}{#2}
       \vspace{-0.60cm}
     \begin{center}
       \begin{small}
         \begin{tabular}{@{}|@{~~}l@{~~}|@{}}
           \hline
           \rule[-1.5ex]{0pt}{1ex}\begin{minipage}[b]{.96\linewidth}
             \vspace{1ex}
             \smallskip
             \begin{center}\FigCaption\end{center}
             }{%
           \end{minipage}\\
           \hline
         \end{tabular}
       \end{small}
       \vspace{-0.2cm}
       \caption{\FigCaption}
       \figlab{\FigLabel}
     \end{center}
   \end{figure}
}

%%%%%%%%%%%%%%%%%%%%%%%%%%%%%%%%%%%%%%%%%%%%%%
% Better Box environment
% Usage: \begin{Boxfig}[placement]{Caption}{Label}{Title}
\newenvironment{Boxfig}[4][]{
    \begin{figure}[thb!]
    \renewcommand{\figurename}{{\bf Figure~}}
    \newcommand{\FigCaption}{#2}
    \newcommand{\FigLabel}{#3}
    \vspace{-0.25cm}
    \begin{center}
        \begin{small}
        \begin{tabular}{@{}|@{~~}l@{~~}|@{}}
        \hline
        \rule[-1.5ex]{0pt}{1ex}\begin{minipage}[b]{.96\linewidth}
        \vspace{1ex}
        \smallskip
        \begin{center}
           \textbf{#4}
        \end{center}
}
{%
        \end{minipage}\\
        \hline
        \end{tabular}
        \end{small}
        \vspace{-0.25cm}
        \caption{\FigCaption}
        \label{fig:\FigLabel}
    \end{center}
    \end{figure}
}
